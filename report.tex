% Options for packages loaded elsewhere
% Options for packages loaded elsewhere
\PassOptionsToPackage{unicode}{hyperref}
\PassOptionsToPackage{hyphens}{url}
\PassOptionsToPackage{dvipsnames,svgnames,x11names}{xcolor}
%
\documentclass[
]{article}
\usepackage{xcolor}
\usepackage{amsmath,amssymb}
\setcounter{secnumdepth}{-\maxdimen} % remove section numbering
\usepackage{iftex}
\ifPDFTeX
  \usepackage[T1]{fontenc}
  \usepackage[utf8]{inputenc}
  \usepackage{textcomp} % provide euro and other symbols
\else % if luatex or xetex
  \usepackage{unicode-math} % this also loads fontspec
  \defaultfontfeatures{Scale=MatchLowercase}
  \defaultfontfeatures[\rmfamily]{Ligatures=TeX,Scale=1}
\fi
\usepackage{lmodern}
\ifPDFTeX\else
  % xetex/luatex font selection
\fi
% Use upquote if available, for straight quotes in verbatim environments
\IfFileExists{upquote.sty}{\usepackage{upquote}}{}
\IfFileExists{microtype.sty}{% use microtype if available
  \usepackage[]{microtype}
  \UseMicrotypeSet[protrusion]{basicmath} % disable protrusion for tt fonts
}{}
\makeatletter
\@ifundefined{KOMAClassName}{% if non-KOMA class
  \IfFileExists{parskip.sty}{%
    \usepackage{parskip}
  }{% else
    \setlength{\parindent}{0pt}
    \setlength{\parskip}{6pt plus 2pt minus 1pt}}
}{% if KOMA class
  \KOMAoptions{parskip=half}}
\makeatother
% Make \paragraph and \subparagraph free-standing
\makeatletter
\ifx\paragraph\undefined\else
  \let\oldparagraph\paragraph
  \renewcommand{\paragraph}{
    \@ifstar
      \xxxParagraphStar
      \xxxParagraphNoStar
  }
  \newcommand{\xxxParagraphStar}[1]{\oldparagraph*{#1}\mbox{}}
  \newcommand{\xxxParagraphNoStar}[1]{\oldparagraph{#1}\mbox{}}
\fi
\ifx\subparagraph\undefined\else
  \let\oldsubparagraph\subparagraph
  \renewcommand{\subparagraph}{
    \@ifstar
      \xxxSubParagraphStar
      \xxxSubParagraphNoStar
  }
  \newcommand{\xxxSubParagraphStar}[1]{\oldsubparagraph*{#1}\mbox{}}
  \newcommand{\xxxSubParagraphNoStar}[1]{\oldsubparagraph{#1}\mbox{}}
\fi
\makeatother


\usepackage{longtable,booktabs,array}
\usepackage{calc} % for calculating minipage widths
% Correct order of tables after \paragraph or \subparagraph
\usepackage{etoolbox}
\makeatletter
\patchcmd\longtable{\par}{\if@noskipsec\mbox{}\fi\par}{}{}
\makeatother
% Allow footnotes in longtable head/foot
\IfFileExists{footnotehyper.sty}{\usepackage{footnotehyper}}{\usepackage{footnote}}
\makesavenoteenv{longtable}
\usepackage{graphicx}
\makeatletter
\newsavebox\pandoc@box
\newcommand*\pandocbounded[1]{% scales image to fit in text height/width
  \sbox\pandoc@box{#1}%
  \Gscale@div\@tempa{\textheight}{\dimexpr\ht\pandoc@box+\dp\pandoc@box\relax}%
  \Gscale@div\@tempb{\linewidth}{\wd\pandoc@box}%
  \ifdim\@tempb\p@<\@tempa\p@\let\@tempa\@tempb\fi% select the smaller of both
  \ifdim\@tempa\p@<\p@\scalebox{\@tempa}{\usebox\pandoc@box}%
  \else\usebox{\pandoc@box}%
  \fi%
}
% Set default figure placement to htbp
\def\fps@figure{htbp}
\makeatother





\setlength{\emergencystretch}{3em} % prevent overfull lines

\providecommand{\tightlist}{%
  \setlength{\itemsep}{0pt}\setlength{\parskip}{0pt}}



 


\makeatletter
\@ifpackageloaded{caption}{}{\usepackage{caption}}
\AtBeginDocument{%
\ifdefined\contentsname
  \renewcommand*\contentsname{Table of contents}
\else
  \newcommand\contentsname{Table of contents}
\fi
\ifdefined\listfigurename
  \renewcommand*\listfigurename{List of Figures}
\else
  \newcommand\listfigurename{List of Figures}
\fi
\ifdefined\listtablename
  \renewcommand*\listtablename{List of Tables}
\else
  \newcommand\listtablename{List of Tables}
\fi
\ifdefined\figurename
  \renewcommand*\figurename{Figure}
\else
  \newcommand\figurename{Figure}
\fi
\ifdefined\tablename
  \renewcommand*\tablename{Table}
\else
  \newcommand\tablename{Table}
\fi
}
\@ifpackageloaded{float}{}{\usepackage{float}}
\floatstyle{ruled}
\@ifundefined{c@chapter}{\newfloat{codelisting}{h}{lop}}{\newfloat{codelisting}{h}{lop}[chapter]}
\floatname{codelisting}{Listing}
\newcommand*\listoflistings{\listof{codelisting}{List of Listings}}
\makeatother
\makeatletter
\makeatother
\makeatletter
\@ifpackageloaded{caption}{}{\usepackage{caption}}
\@ifpackageloaded{subcaption}{}{\usepackage{subcaption}}
\makeatother
\usepackage{bookmark}
\IfFileExists{xurl.sty}{\usepackage{xurl}}{} % add URL line breaks if available
\urlstyle{same}
\hypersetup{
  pdftitle={Economic Development Across Income Groups},
  pdfauthor={Mark Arshavsky (2556041); Chloe Guerrero (2578787); Tiffany Wickens (2537894); Laura Zang (2549139); Chloe Zhao (2593352)},
  colorlinks=true,
  linkcolor={blue},
  filecolor={Maroon},
  citecolor={Blue},
  urlcolor={Blue},
  pdfcreator={LaTeX via pandoc}}


\title{Economic Development Across Income Groups}
\author{Mark Arshavsky (2556041) \and Chloe Guerrero
(2578787) \and Tiffany Wickens (2537894) \and Laura Zang
(2549139) \and Chloe Zhao (2593352)}
\date{}
\begin{document}
\maketitle


\subsection{Introduction}\label{introduction}

Global differences in economic development are a persistent and
consequential feature of the global economy that shape patterns of
opportunity, wellbeing, and inequality. The World Bank offers
classifications that help us examine disparities across countries at
different stages of development. In this project, we analyze data from
the World Development Indicators (WDI) to compare patterns of economic
development across income groups from 2000 to 2023. The project intends
to explore these long-term dynamics.

\subsection{Data Description}\label{data-description}

\paragraph{Data Sources and
Indicators}\label{data-sources-and-indicators}

This project uses data from the World Bank World Development Indicators
(WDI) database. We focus on three core measures of economic development:
GDP per capita (constant USD), annual GDP growth, and the
employment-to-population ratio.

\paragraph{Data Preparation and
Cleaning}\label{data-preparation-and-cleaning}

\begin{itemize}
\item
  The raw WDI data were imported into a SQLite database and reshaped
  from a wide format into a country--year structure. All indicators were
  combined into a unified dataset to support consistent cleaning and
  analysis.
\item
  The dataset was filtered to the 2000--2023 period to ensure
  comparability across indicators. Country and indicator metadata were
  separated from yearly observations to streamline analysis, and
  countries without income group classifications were excluded from
  income-group-based summaries.
\end{itemize}

\paragraph{Income Group
Classification}\label{income-group-classification}

Countries were categorized using World Bank income group
classifications, which were linked to the country--year data using
standard country codes. This framework enables systematic comparison of
economic outcomes across income levels.

\paragraph{Summary of Key Variables}\label{summary-of-key-variables}

\begin{itemize}
\tightlist
\item
  Table 1 summarizes the distribution of GDP per capita, GDP growth, and
  employment ratios across income groups. GDP per capita differs
  substantially across income classifications, while GDP growth and
  employment ratios exhibit more moderate variation.
\end{itemize}

\begin{longtable}[]{@{}llllll@{}}

\caption{\label{tbl-income-summary}Average GDP per capita, GDP growth,
and employment ratios by World Bank income group (2000--2023).}

\tabularnewline

\toprule\noalign{}
& income\_group & n\_country\_years & avg\_emp\_ratio & avg\_gdp\_growth
& avg\_gdp\_pc \\
\midrule\noalign{}
\endhead
\bottomrule\noalign{}
\endlastfoot
0 & High income & 2064 & 57.22 & 2.54 & 35295.24 \\
1 & Upper middle income & 1296 & 53.23 & 3.57 & 5815.63 \\
2 & Lower middle income & 1200 & 55.21 & 4.20 & 2030.59 \\
3 & Low income & 600 & 61.35 & 3.77 & 703.20 \\

\end{longtable}

\begin{itemize}
\tightlist
\item
  Table 2 reports yearly average values of GDP per capita, GDP growth,
  and employment ratios by World Bank income group over the period
  2000--2023.
\end{itemize}

\begin{longtable}[]{@{}llllll@{}}

\caption{\label{tbl-income-yearly}Yearly averages of economic indicators
by income group.}

\tabularnewline

\toprule\noalign{}
& income\_group & year & avg\_emp\_ratio & avg\_gdp\_growth &
avg\_gdp\_pc \\
\midrule\noalign{}
\endhead
\bottomrule\noalign{}
\endlastfoot
0 & High income & 2000 & 56.10 & 4.85 & 29733.48 \\
1 & High income & 2001 & 56.18 & 2.40 & 30204.97 \\
2 & High income & 2002 & 56.03 & 2.36 & 30339.30 \\
3 & High income & 2003 & 55.97 & 3.08 & 30894.56 \\
4 & High income & 2004 & 56.21 & 4.83 & 32013.04 \\

\end{longtable}

\subsection{Data Analysis}\label{data-analysis}

\begin{figure}

\centering{

\pandocbounded{\includegraphics[keepaspectratio]{figures/gdp_employment_pearson_correlations.png}}

}

\caption{\label{fig-gdp-emp-corr}Correlation between GDP growth and
employment-to-population ratios by income group.}

\end{figure}%

\begin{figure}

\centering{

\pandocbounded{\includegraphics[keepaspectratio]{figures/gdp_growth_box.png}}

}

\caption{\label{fig-gdp-growth-box}Distribution of annual GDP growth
rates by income group, 2000--2023.}

\end{figure}%

\begin{figure}

\centering{

\pandocbounded{\includegraphics[keepaspectratio]{figures/gdp_pc_vs_employment_scat.png}}

}

\caption{\label{fig-gdp-pc-employment}Relationship between GDP per
capita and employment-to-population ratios by income group.}

\end{figure}%

\begin{figure}

\centering{

\pandocbounded{\includegraphics[keepaspectratio]{figures/gdp_percapita_by_income.png}}

}

\caption{\label{fig-gdp-pc-income}Average GDP per capita by World Bank
income group, 2000--2023.}

\end{figure}%

\begin{figure}

\centering{

\pandocbounded{\includegraphics[keepaspectratio]{figures/global_indicator_trends.png}}

}

\caption{\label{fig-global-trends}Global mean GDP per capita, GDP
growth, and employment ratios over time.}

\end{figure}%

\begin{figure}

\centering{

\pandocbounded{\includegraphics[keepaspectratio]{figures/indicator_trends_by_income.png}}

}

\caption{\label{fig-trends-by-income}Trends in economic indicators over
time by World Bank income group.}

\end{figure}%

\begin{longtable}[]{@{}llllllllllllll@{}}

\caption{\label{tbl-desc-income}Descriptive statistics for GDP per
capita, GDP growth, and employment by income group.}

\tabularnewline

\toprule\noalign{}
& income\_group & avg\_gdp\_pc\_mean & avg\_gdp\_pc\_median &
avg\_gdp\_pc\_std & avg\_gdp\_pc\_count & avg\_gdp\_growth\_mean &
avg\_gdp\_growth\_median & avg\_gdp\_growth\_std &
avg\_gdp\_growth\_count & avg\_emp\_ratio\_mean &
avg\_emp\_ratio\_median & avg\_emp\_ratio\_std &
avg\_emp\_ratio\_count \\
\midrule\noalign{}
\endhead
\bottomrule\noalign{}
\endlastfoot
0 & High income & 35295.24 & 35295.24 & NaN & 1 & 2.54 & 2.54 & NaN & 1
& 57.22 & 57.22 & NaN & 1 \\
1 & Low income & 703.20 & 703.20 & NaN & 1 & 3.77 & 3.77 & NaN & 1 &
61.35 & 61.35 & NaN & 1 \\
2 & Lower middle income & 2030.59 & 2030.59 & NaN & 1 & 4.20 & 4.20 &
NaN & 1 & 55.21 & 55.21 & NaN & 1 \\
3 & Upper middle income & 5815.63 & 5815.63 & NaN & 1 & 3.57 & 3.57 &
NaN & 1 & 53.23 & 53.23 & NaN & 1 \\

\end{longtable}

\begin{longtable}[]{@{}lllll@{}}

\caption{\label{tbl-desc-yearly}Yearly averages of key economic
indicators aggregated across income groups.}

\tabularnewline

\toprule\noalign{}
& year & avg\_gdp\_pc & avg\_gdp\_growth & avg\_emp\_ratio \\
\midrule\noalign{}
\endhead
\bottomrule\noalign{}
\endlastfoot
0 & 2000 & 9066.7200 & 4.3100 & 57.3500 \\
1 & 2001 & 9199.7850 & 3.4575 & 57.2375 \\
2 & 2002 & 9259.0250 & 3.8800 & 57.0475 \\
3 & 2003 & 9448.0750 & 4.0025 & 56.9925 \\
4 & 2004 & 9807.7050 & 5.7800 & 57.0300 \\
5 & 2005 & 10055.6025 & 5.1075 & 57.0900 \\
6 & 2006 & 10628.8725 & 5.6350 & 57.2550 \\
7 & 2007 & 10981.2725 & 5.5225 & 57.3925 \\
8 & 2008 & 11043.9625 & 4.1150 & 57.3250 \\
9 & 2009 & 10644.8775 & 0.9400 & 56.8300 \\
10 & 2010 & 10840.4850 & 4.7050 & 56.6450 \\
11 & 2011 & 11017.3600 & 3.7050 & 56.7425 \\
12 & 2012 & 11045.4650 & 3.0650 & 56.7150 \\
13 & 2013 & 11156.0475 & 3.1375 & 56.6075 \\
14 & 2014 & 11279.1900 & 3.2150 & 56.5675 \\
15 & 2015 & 11737.8825 & 2.5850 & 56.5700 \\
16 & 2016 & 11524.0525 & 3.0725 & 56.5675 \\
17 & 2017 & 11706.4900 & 3.3300 & 56.5725 \\
18 & 2018 & 11954.5600 & 3.4650 & 56.7050 \\
19 & 2019 & 12117.4825 & 3.1650 & 56.7800 \\
20 & 2020 & 11171.2475 & -4.2175 & 55.0550 \\
21 & 2021 & 11934.8850 & 5.1425 & 55.5350 \\
22 & 2022 & 12352.6750 & 4.2875 & 56.4675 \\
23 & 2023 & 12581.7650 & 3.1200 & 56.9475 \\

\end{longtable}

\subsection{Results \& Discussion}\label{results-discussion}

\subsubsection{Differences in Economic Outcomes Across Income
Groups}\label{differences-in-economic-outcomes-across-income-groups}

Figure~\ref{fig-gdp-pc-income} illustrates substantial differences in
average GDP per capita across World Bank income classifications over the
2000--2023 period. High-income countries exhibit by far the highest
average GDP per capita, exceeding \$35,000 (constant USD), while
upper-middle-income countries average approximately \$6,000.
Lower-middle-income and low-income countries trail significantly behind,
with average GDP per capita below \$3,000 and \$1,000 respectively.
These disparities highlight persistent global inequality in income
levels and economic capacity.

In contrast, employment-to-population ratios show relatively modest
variation across income groups (Table 1). While low-income countries
tend to exhibit slightly higher employment ratios than high-income
countries, the differences are small relative to income gaps. This
suggests that employment rates alone do not capture differences in
economic well-being, as employment in lower-income countries may be
concentrated in lower-productivity and lower-wage activities.

\subsubsection{Global Trends in Economic Indicators Over
Time}\label{global-trends-in-economic-indicators-over-time}

Figure~\ref{fig-global-trends} presents average global trends in GDP per
capita, GDP growth, and employment-to-population ratios. GDP per capita
increases steadily from 2000 through 2019, reflecting sustained global
economic expansion. A sharp decline occurs in 2020, coinciding with the
economic impacts of the COVID-19 pandemic, followed by a partial
recovery in subsequent years.

GDP growth rates display greater year-to-year volatility than GDP per
capita, fluctuating around 2--4\% for most of the sample period. The
most pronounced contraction appears in 2020, when global growth becomes
strongly negative. Employment-to-population ratios remain relatively
stable over time, though a noticeable decline around 2020 indicates
widespread labor market disruptions during the global downturn.

\subsubsection{Income-Group-Specific
Dynamics}\label{income-group-specific-dynamics}

Disaggregating trends by income group reveals important heterogeneity
(Figure~\ref{fig-trends-by-income}). High-income countries consistently
maintain the highest GDP per capita throughout the period, with gradual
long-term growth interrupted mainly by global crises such as the 2008
financial crisis and the COVID-19 pandemic. Upper-middle-income
countries experience steady growth in GDP per capita, modestly narrowing
the gap with high-income economies, although substantial differences
remain.

GDP growth rates across income groups follow similar cyclical patterns,
suggesting that global economic shocks affect countries across
development levels simultaneously. However, growth volatility is more
pronounced in middle- and lower-income countries, indicating greater
macroeconomic vulnerability. Employment-to-population ratios differ more
visibly: low-income countries generally maintain higher employment
ratios, while high-income countries exhibit lower but more stable
employment levels, reflecting differences in demographics, labor market
institutions, and economic structure.

\subsubsection{Interpretation and
Implications}\label{interpretation-and-implications}

Overall, the results suggest that while global economic growth has
raised GDP per capita across all income groups, convergence between
low-income and high-income countries remains limited. Persistent income
gaps indicate that differences in productivity, technological adoption,
and capital accumulation continue to play a central role in shaping
global inequality. The relative similarity of employment rates across
income groups further emphasizes that productivity, rather than labor
force participation, is a key determinant of economic development.

\subsubsection{Limitations}\label{limitations}

This analysis relies on income-group averages, which mask substantial
heterogeneity within groups and across individual countries. In
addition, missing observations for some countries and years may affect
the precision of summary statistics. Future work could extend this
analysis by focusing on regional patterns or conducting country-level
analyses to better understand specific development trajectories.

\subsection{Conclusion}\label{conclusion}

The evidence from 2000-2003 reveals a global economy characterized by
changes in developmental indicators, and we see that structural
inequality across income groups persists. All groups experienced gains
in GDP per capita, but there remains a substantial disparity between
high- and low-income countries. GDP growth rates tend to respond
similarly to major global shocks, which suggests interconnected economic
dynamics, though lower- and middle-income countries face greater
volatility. Employment rates remain relatively stable and more equal,
highlighting that differences in productivity drive most variation in
economic well-being. Thus, we find that global development has advanced,
but longstanding differences in economic capacity continue to shape
countries' development.




\end{document}
